\documentclass{article}
\usepackage{ctex}
\usepackage[left=2cm,right=2cm,top=1.5cm,bottom=1.5cm]{geometry}
\usepackage{amsmath}
\usepackage{tikz}
\usepackage{multirow}
\usepackage{makecell}
\usepackage{float}
\usepackage[thmmarks,amsmath]{ntheorem}
\usepackage{hyperref}
\usepackage{listings}
\usepackage{xcolor}
\usepackage{fontspec}
\setmonofont{Consolas}
\usetikzlibrary{calc}

\begin{document}
	\title{HW 6}
	\author{肖桐 PB18000037}
	\date{2020 年 11 月 1 日}
	\maketitle

	\newtheorem*{solution}{解}

	\begin{solution}\textnormal{\textbf{1.}}
		为$S, L$引入一个继承属性$inh$和一个综合属性$syn$.\newline
		继承属性$inh$表示由当前文法符号推出的字符串的第一个字符, 所在当前句子中的位置.\newline
		综合属性$syn$表示由当前文法符号推出的字符串的最后一位字符, 所在当前句子中的位置.\newline
		可以得到增广文法的语法制导定义为:
		\begin{table}[H]
			\centering
			\caption{语法制导定义}
			\begin{tabular}{|c|c|}
				\hline
				产生式 & 语义规则 \\
				\hline
				$S' \to AS$ & $S.inh = A.syn$ \\
				\hline
				$A \to \varepsilon$ & $A.syn = 0$ \\
				\hline
				$S \to (BL)$ & $B.inh = S.inh + 1$ \\
				& $L.inh = B.syn$ \\
				& $S.syn = L.syn + 1$ \\
				\hline
				$B \to \varepsilon$ & $B.syn = B.inh$ \\
				\hline
				$S \to a$ & $S.syn = a.syn$ \\
				& $printf(S.inh)$ \\
				\hline
				$L \to L_1, CS$ & $L_1.inh = L.inh$ \\
				& $C.inh = L_1.syn + 2$ \\
				& $S.inh = C.syn$ \\
				& $L.syn = S.syn$ \\
				\hline
				$C \to \varepsilon$ & $C.syn = C.inh$ \\
				\hline
				$L \to S$ & $S.inh = L.inh$ \\
				& $L.syn = S.syn$ \\
				\hline
			\end{tabular}
		\end{table}
		转化为栈操作代码为:
		\begin{table}[H]
			\centering
			\caption{栈操作代码}
			\begin{tabular}{|c|c|}
				\hline
				产生式 & 栈操作代码 \\
				\hline
				$S' \to AS$ & $val[top - 1] = val[top]$ \\
				\hline
				$A \to \varepsilon$ & $val[top + 1] = 0$ \\
				\hline
				$S \to (BL)$ & $val[top - 3] = val[top - 1] + 1$ \\
				\hline
				$B \to \varepsilon$ & $val[top + 1] = val[top - 2] + 1$ \\
				\hline
				$S \to a$ & $printf(val[top])$ \\
				\hline
				$L \to L_1, CS$ & $val[top - 3] = val[top]$ \\
				\hline
				$C \to \varepsilon$ & $val[top + 1] = val[top - 2] + 2$ \\
				\hline
				$L \to S$ &  \\
				\hline
			\end{tabular}
		\end{table}
	\end{solution}
	
	\begin{solution}\textnormal{\textbf{2.}}
		(1). 为非终结符$P, D$引入一个综合属性$num$, 用于记录$id$的个数.\newline
		则可以写出语法制导定义如下:
		\begin{table}[H]
			\centering
			\caption{语法制导定义}
			\begin{tabular}{|c|c|}
				\hline
				产生式 & 语义规则 \\
				\hline
				$P \to D$ & $P.num = D.num$ \\
				& $printf(P.num)$ \\
				\hline
				$D \to D_1\ ;\ D_2$ & $D.num = D_1.num + D_2.num$ \\
				\hline
				$D \to id\ :\ T$ & $D.num = 1$ \\
				\hline
				$D \to proc\ id\ ;\ D_1\ ;\ S$ & $D.num = D_1.num + 1$ \\
				\hline
			\end{tabular}
		\end{table}
		(2). 为了打印每一个$id$的嵌套深度, 需要为非终结符$D$增加一个继承属性$depth$, 用于记录嵌套深度.\newline
		可以得到语法制导定义如下:
		\begin{table}[H]
			\centering
			\caption{语法制导定义}
			\begin{tabular}{|c|c|}
				\hline
				产生式 & 语义规则 \\
				\hline
				$P \to D$ & $D.depth = 1$ \\
				\hline
				$D \to D_1\ ;\ D_2$ & $D_1.depth = D.depth + 1$ \\
				& $D_2.depth = D.depth + 1$ \\
				\hline
				$D \to id\ :\ T$ & $printf(D.depth)$ \\
				\hline
				$D \to proc\ id\ ;\ D_1\ ;\ S$ & $D_1.depth = D.depth + 1$ \\
				& $printf(D.depth)$ \\
				\hline
			\end{tabular}
		\end{table}
		对应的翻译方案为:
		\begin{table}[H]
			\centering
			\caption{翻译方案}
			\begin{tabular}{|l|}
				\hline
				$P \to \{D.depth = 1;\}\ D$ \\
				\hline
				$D \to \{D_1.depth = D.depth + 1;\}\ D_1\ ;\ \{D_2.depth = D.depth + 1;\}\ D_2$ \\
				\hline
				$D \to id\ \{printf(D.depth);\}\ :\ T$ \\
				\hline
				$D \to proc\ id\ \{printf(D.depth);\}\ ;\ \{D_1.depth = D.depth + 1;\}\ D_1\ ;\ S$ \\
				\hline
			\end{tabular}
		\end{table}
	\end{solution}
\end{document}