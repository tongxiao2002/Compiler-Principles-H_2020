\documentclass{article}
\usepackage{ctex}
\usepackage[left=2cm,right=2cm,top=1.5cm,bottom=1.5cm]{geometry}
\usepackage{amsmath}
\usepackage{tikz}
\usepackage{multirow}
\usepackage{makecell}
\usepackage{float}
\usepackage[thmmarks,amsmath]{ntheorem}
\usepackage{hyperref}
\usetikzlibrary{positioning, arrows.meta}

\begin{document}
	\title{HW4 - 2}
	\author{肖桐 PB18000037}
	\date{2020 年 10 月 18 日}
	\maketitle

	\newtheorem*{solution}{解}

	\begin{solution}\textnormal{\textbf{1.}}
		(1). 首先为每个产生式编号:\newline
		1. $S' \to S$\newline
		2. $S \to BB$\newline
		3. $B \to bB$\newline
		4. $B \to a$\newline
		根据$GOTO$图构建分析表:
		\begin{table}[H]
			\centering
			\caption{LR分析表}
			\begin{tabular}{|c|ccc|cc|}
				\hline
				\multirow{2}*{状态} & \multicolumn{3}{|c|}{$ACTION$} & \multicolumn{2}{|c|}{$GOTO$} \\
				\cline{2 - 6}
				& $a$ & $b$ & $\$$ & $S$ & $B$ \\
				\hline
				$0$ & $s4$ & $s3$ &  & $1$ & $2$ \\
				$1$ &  &  & $acc$ &  &  \\
				$2$ & $s7$ & $s6$ &  &  & $5$ \\
				$3$ & $s4$ & $s3$ &  &  & $8$ \\
				$4$ & $r4$ & $r4$ &  &  &  \\
				$5$ &  &  & $r2$ &  &  \\
				$6$ & $s7$ & $s6$ &  &  & $9$ \\
				$7$ &  &  & $r4$ &  &  \\
				$8$ & $r3$ & $r3$ &  &  &  \\
				$9$ &  &  & $r3$ &  &  \\
				\hline
			\end{tabular}
		\end{table}
		终结符串$bbabba$的$LR$分析过程为:
		\begin{table}[H]
			\centering
			\caption{处理输入$bbabba$的动作表}
			\begin{tabular}{|l|l|r|c|}
				\hline
				栈 & 符号 & 输入 & 动作 \\
				\hline
				$0$ &  & $bbabba\$$ & 移入 \\
				$03$ & $b$ & $babba\$$ & 移入 \\
				$033$ & $bb$ & $abba\$$ & 移入 \\
				$0334$ & $bba$ & $bba\$$ & 根据$B \to a$归约 \\
				$0338$ & $bbB$ & $bba\$$ & 根据$B \to bB$归约 \\
				$038$ & $bB$ & $bba\$$ & 根据$B \to bB$归约 \\
				$02$ & $B$ & $bba\$$ & 移入 \\
				$026$ & $Bb$ & $ba\$$ & 移入 \\
				$0266$ & $Bbb$ & $a\$$ & 移入 \\
				$02667$ & $Bbba$ & $\$$ & 根据$B \to a$归约 \\
				$02669$ & $BbbB$ & $\$$ & 根据$B \to bB$归约 \\
				$0269$ & $BbB$ & $\$$ & 根据$B \to bB$归约 \\
				$025$ & $BB$ & $\$$ & 根据$S \to BB$归约 \\
				$01$ & $S$ & $\$$ & 接受 \\
				\hline
			\end{tabular}
		\end{table}
		终结符串$bba$的$LR$分析过程为:
		\begin{table}[H]
			\centering
			\caption{处理输入$bba$的动作表}
			\begin{tabular}{|l|l|r|c|}
				\hline
				栈 & 符号 & 输入 & 动作 \\
				\hline
				$0$ &  & $bba\$$ & 移入 \\
				$03$ & $b$ & $ba\$$ & 移入 \\
				$033$ & $bb$ & $a\$$ & 移入 \\
				$0334$ & $bba$ & $\$$ & 报错 \\
				\hline
			\end{tabular}
		\end{table}
		因为$LR(1)$分析器能够确切地将栈顶状态与原串中对应所处的分析位置联系起来, 这样能够避免多余$/$错误的移进、规约操作.\newline
		本题中因为后面紧跟的是$\$$, 而在当前状态下, 只有后面紧跟终结符$b/a$才能进行规约,
		后面紧跟任何终结符都不会进行移进. 分析表中对应$\$$符号动作为空白,
		因此不会做任何多余的规约步骤, 也不会将非法符号移入栈中, 而会马上报错.\newline
		相对的, $LALR(1)$分析器则会将$bba$规约为$B$之后才能发现错误.\newline
		(2). 每个产生式编号与(1)相同, 下面先构建分析表:
		\begin{table}[H]
			\centering
			\caption{LR分析表}
			\begin{tabular}{|c|ccc|cc|}
				\hline
				\multirow{2}*{状态} & \multicolumn{3}{|c|}{$ACTION$} & \multicolumn{2}{|c|}{$GOTO$} \\
				\cline{2 - 6}
				& $a$ & $b$ & $\$$ & $S$ & $B$ \\
				\hline
				$0$ & $s47$ & $s36$ &  & $1$ & $2$ \\
				$1$ &  &  & $acc$ &  &  \\
				$2$ & $s47$ & $s36$ &  &  & $5$ \\
				$36$ & $s47$ & $s36$ &  &  & $89$ \\
				$47$ & $r4$ & $r4$ & $r4$ &  &  \\
				$5$ &  &  & $r2$ &  &  \\
				$89$ & $r3$ & $r3$ & $r3$ &  &  \\
				\hline
			\end{tabular}
		\end{table}
		终结符串$bbabba$的$LR$分析过程为:
		\begin{table}[H]
			\centering
			\caption{处理输入$bbabba$的动作表}
			\begin{tabular}{|l|l|r|c|}
				\hline
				栈 & 符号 & 输入 & 动作 \\
				\hline
				$0$ &  & $bbabba\$$ & 移入 \\
				$0(36)$ & $b$ & $babba\$$ & 移入 \\
				$0(36)(36)$ & $bb$ & $abba\$$ & 移入 \\
				$0(36)(36)(47)$ & $bba$ & $bba\$$ & 根据$B \to a$归约 \\
				$0(36)(36)(89)$ & $bbB$ & $bba\$$ & 根据$B \to bB$归约 \\
				$0(36)(89)$ & $bB$ & $bba\$$ & 根据$B \to bB$归约 \\
				$02$ & $B$ & $bba\$$ & 移入 \\
				$02(36)$ & $Bb$ & $ba\$$ & 移入 \\
				$02(36)(36)$ & $Bbb$ & $a\$$ & 移入 \\
				$02(36)(36)(47)$ & $Bbba$ & $\$$ & 根据$B \to a$归约 \\
				$02(36)(36)(89)$ & $BbbB$ & $\$$ & 根据$B \to bB$归约 \\
				$02(36)(89)$ & $BbB$ & $\$$ & 根据$B \to bB$归约 \\
				$025$ & $BB$ & $\$$ & 根据$S \to BB$归约 \\
				$01$ & $S$ & $\$$ & 接受 \\
				\hline
			\end{tabular}
		\end{table}
		与规范$LR(1)$分析区别最大的地方在于$LALR(1)$对两个$bba$子串移进、归约时所处的状态是相同的,
		而规范$LR(1)$分析对$bba$子串进行移进、归约时所处状态不同.
		这样的结果就是$LALR(1)$分析有时无法仅根据栈顶状态判断当前已经对原串分析到了什么地方.\newline
		比如对上面$bbabba$串的分析, 仅从栈顶状态无法区分此时是对第一个$bba$子串进行分析, 还是对第二个$bba$子串进行分析,
		而必须要进行一定的约归步骤之后才能看得出来. 而规范的$LR(1)$分析因为没有将同心项目合并, 可以根据不同的向前看符号分辨不同的项集, 则不存在这种问题.\newline
		这就导致了$LALR(1)$分析的错误检测能力相较于规范的$LR(1)$分析要差一些.\newline
	\end{solution}
	\begin{solution}]\textnormal{\textbf{2.}}
		(1). 对于文法:
		$$
		S \to aAc
		$$
		$$
		A \to Abb\ |\ b
		$$
		写出其增广文法:
		$$
		S' \to S
		$$
		$$
		S \to aAc
		$$
		$$
		A \to Abb\ |\ b
		$$
		从项$[S' \to S, \$]$构造出整个项集:
		\begin{table}[H]
			\centering
			\caption{}
			\begin{tabular}{|c|c|c|c|}
				\hline
				$I_0$ & $I_1$ & $I_2$ & $I_3$ \\
				\hline
				$S' \to \cdot S, \$$ & $S' \to S\cdot, \$$ & $S \to a\cdot Ac, \$$ & $A \to b\cdot, b/c$ \\
				$S \to \cdot aAc, \$$ &  & $A \to \cdot Abb, b/c$ &  \\
				&  & $A \to \cdot b, b/c$ &  \\
				\hline
			\end{tabular}
		\end{table}
		\begin{table}[H]
			\centering
			\caption{}
			\begin{tabular}{|c|c|c|c|}
				\hline
				$I_4$ & $I_5$ & $I_6$ & $I_7$ \\
				\hline
				$S \to aA\cdot c, \$$ & $S \to aAc\cdot, \$$ & $A \to Ab\cdot b, b/c$ & $A \to Abb\cdot, b/c$ \\
				$A \to A\cdot bb, b/c$ &  &  &  \\
				\hline
			\end{tabular}
		\end{table}
		再对产生式进行编号:\newline
		1. $S' \to S$\newline
		2. $S \to aAc$\newline
		3. $A \to Abb$\newline
		4. $A \to b$\newline
		从而可以得到$LR(1)$分析表:
		\begin{table}[H]
			\centering
			\caption{}
			\begin{tabular}{|c|cccc|cc|}
				\hline
				\multirow{2}*{状态} & \multicolumn{4}{|c|}{$ACTION$} & \multicolumn{2}{|c|}{$GOTO$} \\
				\cline{2 - 7}
				& $a$ & $b$ & $c$ & $\$$ & $S$ & $A$ \\
				\hline
				$0$ & $s2$ &  &  &  & $1$ &  \\
				$1$ &  &  &  & $acc$ &  &  \\
				$2$ &  & $s3$ &  &  &  & $4$ \\
				$3$ &  & $r4$ & $r4$ &  &  &  \\
				$4$ &  &  & $s5$ &  &  &  \\
				$5$ &  & $s6$ &  & $r2$ &  &  \\
				$6$ &  & $s7$ &  &  &  &  \\
				$7$ &  & $r3$ & $r3$ &  &  &  \\
				\hline
			\end{tabular}
		\end{table}
		可见该分析表无冲突, 因此该文法是$LR(1)$文法.\newline
		(2). 对于文法:
		$$
		S \to aAc
		$$
		$$
		A \to bAb\ |\ b
		$$
		写出其增广文法:
		$$
		S' \to S
		$$
		$$
		S \to aAc
		$$
		$$
		A \to bAb\ |\ b
		$$
		从项$[S' \to S, \$]$构造出整个项集:
		\begin{table}[H]
			\centering
			\caption{}
			\begin{tabular}{|c|c|c|c|c|c|}
				\hline
				$I_0$ & $I_1$ & $I_2$ & $I_3$ & $I_4$ & $I_5$ \\
				\hline
				$S' \to \cdot S, \$$ & $S' \to S\cdot, \$$ & $S \to a\cdot Ac, \$$ & $A \to b\cdot Ab, c$ & $A \to b\cdot Ab, b$ & $A \to bA\cdot b, c$ \\
				$S \to \cdot aAc, \$$ &  & $A \to \cdot bAb, c$ & $A \to b\cdot, c$ & $A \to b\cdot, b$ &  \\
				&  & $A \to \cdot b, c$ & $A \to \cdot bAb, b$ & $A \to \cdot bAb, b$ &  \\
				&  &  & $A \to \cdot b, b$ & $A \to \cdot b, b$ &  \\
				\hline
			\end{tabular}
		\end{table}
		\begin{table}[H]
			\centering
			\caption{}
			\begin{tabular}{|c|c|c|c|c|}
				\hline
				$I_6$ & $I_7$ & $I_{8}$ & $I_{9}$ & $I_{10}$ \\
				\hline
				$A \to bAb\cdot, c$ & $A \to bA\cdot b, b$ & $A \to bAb\cdot, b$ & $S \to aA\cdot c, \$$ & $S \to aAc\cdot, \$$ \\
				\hline
			\end{tabular}
		\end{table}
		再对产生式进行编号:\newline
		1. $S' \to S$\newline
		2. $S \to aAc$\newline
		3. $A \to bAb$\newline
		4. $A \to b$\newline
		从而可以得到$LR(1)$分析表:
		\begin{table}[H]
			\centering
			\caption{}
			\begin{tabular}{|c|cccc|cc|}
				\hline
				\multirow{2}*{状态} & \multicolumn{4}{|c|}{$ACTION$} & \multicolumn{2}{|c|}{$GOTO$} \\
				\cline{2 - 7}
				& $a$ & $b$ & $c$ & $\$$ & $S$ & $A$ \\
				\hline
				$0$ & $s2$ &  &  &  & $1$ &  \\
				$1$ &  &  &  & $acc$ &  &  \\
				$2$ &  & $s3$ &  &  &  & $9$ \\
				$3$ &  & $s4$ & $r4$ &  &  & $5$ \\
				$4$ &  & $s4, r4$ &  &  &  & $7$ \\
				$5$ &  & $s6$ &  &  &  &  \\
				$6$ &  &  & $r3$ &  &  &  \\
				$7$ &  & $s8$ &  &  &  &  \\
				$8$ &  & $r3$ &  &  &  &  \\
				$9$ &  &  & $s10$ &  &  &  \\
				$10$ &  &  &  & $r2$ &  &  \\
				\hline
			\end{tabular}
		\end{table}
		可见该分析表在$(4, b)$处发生移进-约归冲突, 因此该文法是$LR(1)$文法. 发生冲突的项集为$I_4$.
	\end{solution}
\end{document}