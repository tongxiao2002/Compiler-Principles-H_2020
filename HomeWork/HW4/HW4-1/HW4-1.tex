\documentclass{article}
\usepackage{ctex}
\usepackage[left=2cm,right=2cm,top=1.5cm,bottom=1.5cm]{geometry}
\usepackage{amsmath}
\usepackage{tikz}
\usepackage{multirow}
\usepackage{makecell}
\usepackage{float}
\usepackage[thmmarks,amsmath]{ntheorem}
\usepackage{hyperref}
\usetikzlibrary{positioning, arrows.meta}

\begin{document}
	\title{HW4 - 1}
	\author{肖桐 PB18000037}
	\date{2020 年 10 月 17 日}
	\maketitle
	
	\tikzset
	{
		rect1/.style = {
    	shape = rectangle,
    	draw = black,
    	text width = 2.4cm,
    	align = left,
    	minimum height = 1cm,
		}
	}
	\tikzset
	{
		rect2/.style = {
    	shape = rectangle,
    	draw = black,
    	text width = 4cm,
    	align = left,
    	minimum height = 1cm,
		}
	}
	\newtheorem*{solution}{解}

	\begin{solution}\textnormal{\textbf{1.}}
		(1). SLR分析表:\newline
		首先写出对应的增广文法:\newline
		$$
		E' \to E
		$$
		$$
		E \to E + T\ |\ T
		$$
		$$
		T \to TF\ |\ F
		$$
		$$
		F \to F*\ |\ a\ |\ b
		$$
		令$I_0 = \{E' \to E\}$, 进而得到$GOTO$图如下:\newline
		\begin{tikzpicture}
			\node[rect1](0) at (0, 0)
			{
				$I_0$\newline
				$E' \to \cdot E$\newline
				$E \to \cdot E + T$\newline
				$E \to \cdot T$\newline
				$T \to \cdot TF$\newline
				$T \to \cdot F$\newline
				$F \to \cdot F*$\newline
				$F \to \cdot a$\newline
				$F \to \cdot b$
			};
			\node[rect1](1) at (3.5cm, 1.7cm)
			{
				$I_1$\newline
				$E' \to E\cdot$\newline
				$E \to E \cdot + T$
			};
			\node[rect1, below = 1cm of 0](2)
			{
				$I_2$\newline
				$E \to T\cdot$\newline
				$T \to T\cdot F$\newline
				$F \to \cdot F*$\newline
				$F \to \cdot a$\newline
				$F \to \cdot b$
			};
			\node[rect1, below = 1cm of 1](3)
			{
				$I_3$\newline
				$T \to F\cdot$\newline
				$F \to F\cdot *$
			};
			\node[rect1, below = 1cm of 3](4)
			{
				$I_4$\newline
				$F \to a\cdot$
			};
			\node[rect1, below = 1cm of 4](5)
			{
				$I_5$\newline
				$F \to b\cdot$
			};
			\node[rect1](6) at (7cm, 0.6cm)
			{
				$I_6$\newline
				$E \to E + \cdot T$\newline
				$T \to \cdot TF$\newline
				$T \to \cdot F$\newline
				$F \to \cdot F*$\newline
				$F \to \cdot a$\newline
				$F \to \cdot b$
			};
			\node[rect1, below = 1cm of 6](7)
			{
				$I_7$\newline
				$F \to F*\cdot$
			};
			\node[rect1, below = 1cm of 2](8)
			{
				$I_8$\newline
				$T \to TF\cdot$\newline
				$F \to F\cdot *$
			};
			\node[rect1, right = 1cm of 6](9)
			{
				$I_9$\newline
				$E \to E + T\cdot$\newline
				$T \to T\cdot F$\newline
				$F \to \cdot F*$\newline
				$F \to \cdot a$\newline
				$F \to \cdot b$
			};
			\draw[->] (0) -- node[above]{$E$}(1);
			\draw[->] (0) -- node[right]{$T$}(2);
			\draw[->] (0) -- node[above]{$F$}(3);
			\draw[->] (0) -- node[above]{$a$}(4);
			\draw[->] (0) -- (1.5cm, -2cm) -- node[right]{$b$}(1.5cm, -4.8cm) -- (5);
			\draw[->] (1) -- node[above]{$+$}(6);
			\draw[->] (6) -- node[above]{$T$}(9);
			\draw[->] (3) -- node[above]{$*$}(7);
			\draw[->] (2) -- node[right]{$F$}(8);
			\draw[->] (6) -- (5.1cm, 0cm) -- node[left]{$a$}(5.1cm, -3cm) -- (4);
			\draw[->] (6) -- (5.3cm, -1cm) -- node[right]{$b$}(5.3cm, -5cm) -- (5);
			\draw[->] (2) -- (1.9cm, -5.4cm) -- node[right]{$a$}(1.9cm, -3.5cm) -- (4);
			\draw[->] (2) -- (1.5cm, -6cm) -- node[below]{$b$}(5);
			\draw[->] (8) -- node[above]{$*$}(7cm, -8.8cm) -- (7);
			\draw[->] (9) -- node[left]{$a$}(10.6cm, -4cm) -- (4);
			\draw[->] (9) -- node[right]{$b$}(11cm, -5cm) -- (5);
			\draw[->] (9) -- node[right]{$F$}(12cm, -9.5cm) -- (8);
			\draw[->] (6) -- (3.6cm, 0.6cm) -- node[left]{$F$}(3);
		\end{tikzpicture}\newline
		再对产生式进行编号:\newline
		1. $E' \to E$\newline
		2. $E \to E + T$\newline
		3. $E \to T$\newline
		4. $T \to TF$\newline
		5. $T \to F$\newline
		6. $F \to F*$\newline
		7. $F \to a$\newline
		8. $F \to b$\newline
		则可以得到$SLR$分析表如下:\newline
		\begin{table}[H]
			\centering
			\caption{}
			\begin{tabular}{|c|ccccc|ccc|}
				\hline
				\multirow{2}*{状态} & \multicolumn{5}{|c|}{$ACTION$} & \multicolumn{3}{|c|}{$GOTO$} \\
				\cline{2 - 9}
									& $+$ & $*$ & $a$ & $b$ & $\$$ & $E$ & $T$ & $F$ \\
				\hline
				$0$ &  &  & $s4$ & $s5$ &  & $1$ & $2$ & $3$ \\
				\hline
				$1$ & $s6$ &  &  &  & $acc$ &  &  & \\
				\hline
				$2$ & $r3$ &  & $s4$ & $s5$ & $r3$ &  &  & $8$ \\
				\hline
				$3$ & $r5$ & $s7$ & $r5$ & $r5$ & $r5$ &  &  & \\
				\hline
				$4$ & $r7$ & $r7$ & $r7$ & $r7$ & $r7$ &  &  & \\
				\hline
				$5$ & $r8$ & $r8$ & $r8$ & $r8$ & $r8$ &  &  & \\
				\hline
				$6$ &  &  & $s4$ & $s5$ &  &  & $9$ & $3$ \\
				\hline
				$7$ & $r6$ & $r6$ & $r6$ & $r6$ & $r6$ &  &  & \\
				\hline
				$8$ & $r4$ & $s7$ & $r4$ & $r4$ & $r4$ &  &  & \\
				\hline
				$9$ & $r2$ &  & $s4$ & $s5$ & $r2$ &  &  & $8$ \\
				\hline
			\end{tabular}
		\end{table}
		(2). LALR分析表:\newline
		先写出(1)中每个项集的内核项:\newline
		$I_0:\ E' \to \cdot E$\newline
		$I_1:\ E' \to E\cdot,\ E \to E\cdot + T$\newline
		$I_2:\ E \to T\cdot,\ T \to T\cdot F$\newline
		$I_3:\ T \to F\cdot,\ F \to F\cdot *$\newline
		$I_4:\ F \to a\cdot$\newline
		$I_5:\ F \to b\cdot$\newline
		$I_6:\ E \to E + \cdot T$\newline
		$I_7:\ F \to F*\cdot$\newline
		$I_8:\ T \to TF\cdot,\ F \to F\cdot *$\newline
		$I_9:\ E \to E + T\cdot,\ T \to T\cdot F$\newline
		下面为这些内核项确定向前看符号:\newline
		通过传播和自发生成过程可以得到各个内核项的向前看符号如下:\newline
		$I_0:\ E' \to \cdot E, \$$\newline
		$I_1:\ E' \to E\cdot, \$;\ E \to E \cdot + T, \$/+$\newline
		$I_2:\ E \to T\cdot, \$/+;\ T \to T \cdot F, \$/+/a/b$\newline
		$I_3:\ T \to F\cdot, \$/+/a/b;\ T \to F\cdot *, \$/+/*/a/b$\newline
		$I_4:\ F \to a\cdot, \$/+/*/a/b$\newline
		$I_5:\ F \to b\cdot, \$/+/*/a/b$\newline
		$I_6:\ E \to E +\cdot T, \$/+$\newline
		$I_7:\ F \to F*\cdot, \$/+/*/a/b$\newline
		$I_8:\ T \to TF\cdot, \$/+/a/b;\ F \to F\cdot *, \$/+/*/a/b$\newline
		$I_9:\ E \to E + T\cdot, \$/+;\ T \to T\cdot F, \$/+/a/b$\newline
		对每个内核项作闭包可得项目集如下:
		\begin{table}[H]
			\centering
			\caption{}
			\begin{tabular}{|c|c|c|c|c|}
				\hline
				$I_0$ & $I_1$ & $I_2$ & $I_3$ & $I_4$ \\
				\hline
				$E' \to \cdot E, \$$ & $E' \to E\cdot, \$$ & $E \to T\cdot, \$/+$ & $T \to F\cdot, \$/+/a/b$ & $F \to a\cdot, \$/+/*/a/b$ \\

				$E \to \cdot E + T, \$/+$ & $E \to E\cdot +T, \$/+$ & $T \to T\cdot F, \$/+/a/b$ & $T \to F\cdot *, \$/+/*/a/b$ & \\
				
				$E \to \cdot T, \$/+$ &  & $F \to \cdot F*, */a/b$ &  & \\
				
				$T \to \cdot TF, \$/+/a/b$ &  & $F \to \cdot a, */a/b$ &  & \\
				
				$T \to \cdot F, \$/+/a/b$ &  & $F \to \cdot b, */a/b$ &  & \\
				
				$F \to \cdot F*, \$/+/*/a/b$ &  &  &  & \\
				
				$F \to \cdot a, \$/+/*/a/b$ &  &  &  & \\
				
				$F \to \cdot b, \$/+/*/a/b$ &  &  &  & \\
				\hline
			\end{tabular}
		\end{table}
		\begin{table}[H]
			\centering
			\caption{}
			\begin{tabular}{|c|c|c|c|}
				\hline
				$I_5$ & $I_6$ & $I_7$ & $I_8$ \\
				\hline
				$F \to b\cdot, \$/+/*/a/b$ & $E \to E+\cdot T, \$/+$ & $F \to F*\cdot, \$/+/*/a/b$ & $T \to TF\cdot, \$/+/a/b$ \\
				
				& $T \to \cdot TF, \$/+/a/b$ &  & $F \to F\cdot *, \$/+/*/a/b$ \\
				
				& $T \to \cdot F, \$/+/a/b$ &  & \\
				
				& $F \to \cdot F*, \$/+/*/a/b$ &  & \\
				
				& $F \to \cdot a, \$/+/*/a/b$ &  & \\
				
				& $F \to \cdot b, \$/+/*/a/b$ &  & \\
				
				&  &  & \\
				
				&  &  & \\
				\hline
			\end{tabular}
		\end{table}
		\begin{table}[H]
			\centering
			\caption{}
			\begin{tabular}{|c|}
				\hline
				$I_9$\\
				\hline
				$E \to E + T\cdot, \$/+$ \\
				
				$T \to T\cdot F, \$/+/a/b$\\
				
				$F \to \cdot F*, \$/+/*/a/b$\\
				
				$F \to \cdot a, \$/+/*/a/b$\\
				
				$F \to \cdot b, \$/+/*/a/b$\\
				\hline
			\end{tabular}
		\end{table}
		产生式编号与(1)相同, 故得到$LALR(1)$分析表如下:
		\begin{table}[H]
			\centering
			\caption{}
			\begin{tabular}{|c|ccccc|ccc|}
				\hline
				\multirow{2}*{状态} & \multicolumn{5}{|c|}{$ACTION$} & \multicolumn{3}{|c|}{$GOTO$} \\
				\cline{2 - 9}
				& $+$ & $*$ & $a$ & $b$ & $\$$ & $E$ & $T$ & $F$ \\
				\hline
				$0$ &  &  & $s4$ & $s5$ &  & $1$ & $2$ & $3$ \\
				\hline
				$1$ & $s6$ &  &  &  & $acc$ &  &  &  \\
				\hline
				$2$ & $r3$ &  & $s4$ & $s5$ & $r3$ &  &  & $8$ \\
				\hline
				$3$ & $r5$ & $s7$ & $r5$ & $r5$ & $r5$ &  &  &  \\
				\hline
				$4$ & $r7$ & $r7$ & $r7$ & $r7$ & $r7$ &  &  &  \\
				\hline
				$5$ & $r8$ & $r8$ & $r8$ & $r8$ & $r8$ &  &  &  \\
				\hline
				$6$ &  &  & $s4$ & $s5$ &  &  & $9$ & $3$ \\
				\hline
				$7$ & $r6$ & $r6$ & $r6$ & $r6$ & $r6$ &  &  &  \\
				\hline
				$8$ & $r4$ & $s7$ & $r4$ & $r4$ & $r4$ &  &  &  \\
				\hline
				$9$ & $r2$ &  & $s4$ & $s5$ & $r2$ &  &  & $8$ \\
				\hline
			\end{tabular}
		\end{table}
	\end{solution}
	\begin{solution}\textnormal{\textbf{2.}}
		因为$FIRST(SA) = FIRST(A) = \{a\}$, 因此$FIRST(SA) \bigcap FIRST(A) = \{a\} \neq \emptyset$\newline
		故该文法不是$LL(1)$文法.\newline
		下面构造$LR(1)$分析表证明该文法是$LR(1)$文法.\newline
		首先构造增广文法:
		$$
		S' \to S
		$$
		$$
		S \to SA\ |\ A
		$$
		$$
		A \to a
		$$
		然后从项$[S' \to S, \$]$构造出整个项集:
		\begin{table}[H]
			\centering
			\caption{}
			\begin{tabular}{|c|c|c|c|c|}
				\hline
				$I_0$ & $I_1$ & $I_2$ & $I_3$ & $I_4$ \\
				\hline
				$S' \to \cdot S, \$$ & $S' \to S\cdot, \$$ & $S \to SA\cdot, \$/a$ & $S \to A\cdot, \$/a$ & $A \to a\cdot, \$/a$ \\
				$S \to \cdot SA, \$/a$ & $S \to S\cdot A, \$/a$ &  &  &  \\
				$S \to \cdot A, \$/a$ & $A \to \cdot a, \$/a$ &  &  &  \\
				$A \to \cdot a, \$/a$ &  &  &  &  \\
				\hline
			\end{tabular}
		\end{table}
		对各个产生式进行编号:\newline
		1. $S' \to S$\newline
		2. $S \to SA$\newline
		3. $S \to A$\newline
		4. $A \to a$\newline
		因此对应的$LR(1)$分析表为:
		\begin{table}[H]
			\centering
			\caption{}
			\begin{tabular}{|c|cc|cc|}
				\hline
				\multirow{2}*{状态} & \multicolumn{2}{|c|}{$ACTION$} & \multicolumn{2}{|c|}{$GOTO$} \\
				\cline{2 - 5}
				& $a$ & $\$$ & $S$ & $A$ \\
				\hline
				$0$ & $s4$ &  & $1$ & $3$ \\
				$1$ & $s4$ & $acc$ &  & $2$ \\
				$2$ & $r2$ & $r2$ &  &  \\
				$3$ & $r3$ & $r3$ &  &  \\
				$4$ & $r4$ & $r4$ &  &  \\
				\hline
			\end{tabular}
		\end{table}
		即分析表无冲突, 因此该文法是$LR(1)$文法.
	\end{solution}
\end{document}